\documentclass{article}
\usepackage{graphicx,tikz}
\usepackage{SIunits}
\usepackage[graphics,tightpage,active]{preview}
\PreviewEnvironment{tikzpicture}
\newlength\imagewidth
\newlength\imagescale
\begin{document}

\pgfmathsetlength{\imagewidth}{\linewidth} % desired displayed width of image
\pgfmathsetlength{\imagescale}{\imagewidth/670} % pixel width of image
\usetikzlibrary{shapes.arrows}
%
%% adjust scale of tikzpicture (and direction of y) such that pixel
%% coordinates can be used for drawing overlays:
\begin{tikzpicture}[x=\imagescale,y=-\imagescale]

% place image (integer coordinates refer to pixel centers):
\node[anchor=north west,inner sep=0pt,outer sep=0pt] at (0,0)
  {\includegraphics[width=\imagewidth]{segment-crop0041}};

%\draw (100,500) node ;
\draw[|-|,thick] (25,500) -- (175,500) node[midway,above] {\unit{300}{\micro\meter}};
\end{tikzpicture}


%\begin{tikzpicture}
%\draw [help lines] (0,0) grid (3,2);
%\coordinate (a) at (1,0);
%\coordinate (b) at (3,1);
%\draw (a) -- (b);
%\coordinate (c) at ($ (a)!.25!(b) $);
%\coordinate (d) at ($ (c)!1cm!90:(b) $);
%\draw [<->] (c) -- (d) node [sloped,midway,above] {1cm};
%\end{tikzpicture}

\end{document}
